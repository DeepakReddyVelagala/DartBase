\documentclass[12pt]{article}

% Packages
\usepackage[utf8]{inputenc}
\usepackage[T1]{fontenc}
\usepackage{lipsum} % For generating dummy text
\usepackage[margin=2cm]{geometry} % Adjust the margin size here
\usepackage{amsmath}


% my formulae
\newcommand{\bold}[1]{\textbf{#1}}



% Title and author
\title{NLP CS5803 IITH Notes}
\author{Deepak}

\begin{document}

\maketitle
\tableofcontents

\newpage

\section{Introduction}
The starting question is how to make computers understand human
language. We need to find smart ways of representing the language.

\section{Input Representation}
    Consider text modality, where the input is a document, it
    consists of words(also called tokens). The document is represented
    by the set of words/tokens it contains.

    Lets see some methods for text representation
    

    \subsection{TF-IDF scheme}

        \subsubsection{Formulation}
            The TF-IDF (Term Frequency-Inverse Document Frequency) scheme is a
            popular technique used to represent the importance of words in a 
            document corpus. \\
            It combines two factors: term frequency (TF) and inverse document frequency (IDF). \\
            \bold{TF} measures the frequency of a term in a document. 
            It is calculated by counting the number of occurrences of a term 
            in a document as a raw or by taking a log of it.

            \[
            \text{tf}(t, d) = \begin{cases}
                            0 & \text{if } c(t, d) = 0 \\
                            1 + \log(c(t, d)) & \text{if } c(t, d) \neq 0
                        \end{cases}
            \]

            where $c(t, d)$ is the count of term $t$ in document $d$.
            \\
            \bold{IDF} de-emphasizes the frequent words across the corpus
            (all documents combined is usually called corpus) and emphasizes
            the on words differentiating the documents.

            \[
            \text{idf}(t) = \log_{10} \left( \frac{N}{\text{df}_{t}}\right)
            \]
            
            where $N$ is the total number of documents in the corpus and $\text{df}_{t}$ is the number of documents containing the term $t$.
            \\
            The TF-IDF score for a term in a document is obtained by multiplying its TF value with its IDF value. Mathematically, it can be represented as:

            \[
            \text{tf-idf}(t, d) = \text{tf}(t, d) \times \text{idf}(t)
            \]
            Now we get a table with TF-IDF values.
            
            This way, each document is represented by a vector from the column
            of the table and each word is presented by a vector from
            the row of the table.

            \subsubsection{Limitation}
            \begin{itemize}
                \item Words are considered at their lexical appearance
                \item Synonymy is not considered
                \item Polysemy(word with multiple meanings) is not considered
                \item long sparse vectors
                \item context is not considered
            \end{itemize}

    \subsection{SVD - text representation}
        \subsubsection{Formulation}
            Consider the term-document matrix and apply SVD to it
            to do dimensionality reduction, so that we can get dense
            representations.
            \\
            Let A be the term-document matrix(matrix with freq. of words in docs), 
            where each row represents a term and each column represents a document, 
            by SVD we have

            \[
                A = U \Sigma V^{T}
            \]
            U, V are orthonormal matrices and $\Sigma$ is a diagonal matrix of 
            singular values of A in decreasing order.
            \\
            By keep only the first k singular values, we have

            \[
                A_k = U_k \Sigma_k V_k^{T}
            \]

            The k here is much smaller than the original dimension of A.
            Terms can be represented by the rows of $U_k$ and documents can be
            represented by the columns of $V_k$.
            
    \subsection{LDA - text representation}
        \subsubsection{Formulation}
            LDA (Latent Dirichlet Allocation) is a text representation based
            on \bold{topics}. It assumes that each document is a mixture of 
            topics which are latent or unknown. Words in a document depend on
            topics of the document. LDA is a mechanism to identify the topics
            and connect words with topics
            \\
            The generative process is as follows:
            \begin{itemize}
                \item For each document, draw a distribution over topics
                \item For each word in the document, draw a topic from the
                distribution over topics and then draw a word from the
                distribution over words for that topic.
            \end{itemize}
            The parameters of the model are the topic distributions for each
            document, the word distributions for each topic, and the topic
            distribution over the entire corpus.
            \\
            The model is trained by maximizing the likelihood of the observed
            documents. The topic distributions for each document and the word
            distributions for each topic are learned from the data.
            \\
            The learned topic distributions for each document can be used to
            represent the documents and the learned word distributions for each
            topic can be used to represent the topics.
            
        \subsubsection{Mathematical Formulation}

            The mathematical formulation of LDA is as follows:
            
            For each document $d$ in the corpus $D$:
            
            1. Choose $N \sim Poisson(\xi)$.
            
            2. Choose $\theta \sim Dir(\alpha)$.
            
            3. For each of the $N$ words $w$:
            
                a. Choose a topic $z \sim Multinomial(\theta)$.
            
                b. Choose a word $w$ from $p(w|z,\beta)$, a multinomial probability conditioned on the topic $z$.
            
            Here, $\xi$ is the parameter of the Poisson distribution used to choose the number of words in a document, $\alpha$ is the parameter of the Dirichlet distribution used to generate the per-document topic distributions, and $\beta$ is the parameter of the multinomial distribution used to generate the per-topic word distribution.
            
            

\section{Results}
\lipsum[5-6] % Dummy text

\section{Conclusion}
\lipsum[7] % Dummy text

\end{document}
